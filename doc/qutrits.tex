Most quantum processors are currently based on qubits.
Quantum computing based on $3$-level systems is known to offer many advantages over qubit-based quantum computing.
As an example, error correction can be done more efficiently \cite{campbell14} and quantum cryptography is more robust \cite{bechmann00}.
A qutrit processor has also recently been used to thermalization in closed quantum systems \cite{blok20}.
In this chapter I will introduce the reader to the basic concepts of $3$-level systems and discuss the generalization of the algorithm discussed in chapter 4 to qutrit systems.\\
First, we will look at the generalization of the Bloch representation to $d$-level systems.
We represent a state $\rho$ with the help of a $d^2-1$-dimensional Bloch vector $\bm{\tau}$.
We call the Bloch space the space of states that fulfill the conditions presented in chapter 1.
A state $\rho$ can be represented in the following way:
$$\rho = \frac{1}{d} \1 + \sum_{i=1}^{d^2-1} \tau_i \sigma_i$$
where $\sigma_i$ are generators of $SU(d)$ obeying
\[
	\sigma_i\sigma_j = \frac{2}{d}\delta_{ij} + d_{ijk}\sigma_k + if_{ijk}\sigma_k
.\]

the $f_{ijk}$ and $d_{ijk}$ are the structure constants of the Lie-Algebra.
$f_{ijk}$ is totally antisymmetric and equals the Levi-Civita-Symbol for $d=2$, $d_{ijk}$ is totally symmetric and vanishing for $d=2$.
We can construct the generators as follows:\cite{kimura03}
 \[
\{\sigma_i\}^{d^2-1}_{i=1} = \{u_{jk},v_{jk},w_l\}
.\]
where
$$
	u_{jk} = \ket{k}\bra{k}+\ket{k}\bra{j}, ~ v_{jk} = -i(\ket{j}\bra{k}-\ket{k}\bra{j}),
$$
$$
	w_l = \sqrt{\frac{2}{l(l+1}} \sum_{j=1}^{l} \left( \ket{j}\bra{j}-l\ket{l+1}\bra{l+1} \right),$$
	$$ 1\le j\le k\le d, 1\le l\le d-1$$

The $\tau_i$ are the components of the bloch vector and are the expectation values of the $\sigma_i$:
$$
	 \tau_i = Tr(\rho\sigma_i)
$$
For $M\ge3$ there are bloch vectors with $\left|\tau\right|\le 1$ which do not correspond to a positive semi-definite matrix.
The space spanned by the bloch-vectors is therefore not a solid ball with radius $1$.
The generators of  $SU(3)$ are the Gell-Mann-matrices. \todo{tex gellmann} \todo{more math on qudits}\\
