\todo{elaborate on relaxation, general stuff like duals, solvability etc}
\todo{describe classical maxcut sdp relaxation algorithm, goemans}
In semidefinite programing we try to maximize a linear function, such that an affine combination of symmetric matrices is positive semidefinite. \cite{vandenberghe96}
An affine combination is a linear combination $\sum_{i=1}^{n}a_ix_i$
 where $x_i$ are elements of a vector space, such that $\sum_{i=1}^{n}a_i=1$
Semidefinite programs are very useful, as they can be solved efficiently both in theory and in practice.\\
For approximating the optimal solution to the max-cut problem, Goemans and Williamson \cite{goemans95} start with reformulating the problem itself: \begin{align*}
	\text{Maximize}\ \frac{1}{2}\sum_{i<j} w_{i,j}\left( 1-y_iy_j \right) \\
	\text{subject to:}\ y_i \in S=\{-1,1\}\, \forall i \in V
\end{align*}\todo{fix notation and definitions}
given a vertex set $ V=\{1,\ldots n\} $ and non-negative weights $w_{i,j}=w_{j,i}$.
We call this a quadratic program.
As this is in NP, we need to relax the constraints.
This is accomplished by extending the objective function to a larger space, namely $S^{n-1}= \{-1,1\}^{n-1}$.
We then have to consider vectors $v_i$ and look at the inner product  $v_i\cdot v_j$
The program therefore changes accordingly:
 \begin{align*}
	\label{eq:}
	\text{Maximize}\ \frac{1}{2}\sum_{i<j} w_{i,j}\left( 1-v_iv_j \right) \\
	\text{subject to:}\ v_i \in S^n=\{-1,1\}^n\, \forall i \in V
\end{align*}
The algorithm that approximates the solution is:
\begin{enumerate}
	\item Solve the relaxed semidefinite program, obtaining an optimal set of vectors $v_i$
	\item Let $r$ be a vector uniformly distributed on the unit sphere
	\item Set the cut to $S=\{i|v_i\cdot r \ge 0\} $
\end{enumerate}
We can build a good geometrical intuition for this.
Think of the vector $r$ as the normal of a hyperplane through the origin, and partition the vertices into ones that lie below it and above it.
This algorithm has an approximation rate of $0.878$.
