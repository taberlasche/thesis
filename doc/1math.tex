The aim of this thesis is to introduce the reader to quantum approximation algorithms in general, with special focus on the algorithm by Bravyi et al. \cite{bravyi19}.
Therefore, I will start with an introduction to the necessary mathematical concepts and to product state approximation.
Furthermore, \todo{more introductory fluff}
In general, a complex $M \times M$ matrix is a density matrix if it is:
\begin{center}\begin{enumerate}
	\item Hermitian, $\rho =\rho^{\dagger}$
	\item positive,	$\rho \ge 0$
	\item normalized, $Tr\rho = 1$
\end{enumerate}\end{center}\todo{better typesetting}
The set of density matrices is convex set and its extremal points are the pure states obeying $\rho^2 = \rho$
We can write a state $\rho$ as
$$\rho = \frac{1}{M} \1 + \sum_{i=1}^{M^2-1} \tau_i \sigma_i$$
where $\sigma_i$ are generators of $SU(M)$ obeying
\[
	\sigma_i\sigma_j = \frac{2}{M}\delta_{ij} + d_{ijk}\sigma_k + if_{ijk}\sigma_k
.\]
$f_{ijk}$ is totally antisymmetric and equals the Levi-Civita-Symbol for $M=2$, $d_{ijk}$ is totally symmetric and vanishing for $M=2$.
This is the Bloch representation of quantum states.
We can construct the generators as follows:\cite{kimura03}
 \[
\{\sigma_i\}^{M^2-1}_{i=1} = \{u_{jk},v_{jk},w_l\}
.\]
where
$$
	u_{jk} = \ket{k}\bra{k}+\ket{k}\bra{j}, ~ v_{jk} = -i(\ket{j}\bra{k}-\ket{k}\bra{j}),
$$
$$
	w_l = \sqrt{\frac{2}{l(l+1}} \sum_{j=1}^{l} \left( \ket{j}\bra{j}-l\ket{l+1}\bra{l+1} \right),$$
	$$ 1\le j\le k\le M, 1\le l\le M-1$$
The $\tau_i$ are the components of the $M^2-1$ dimensional bloch vector and are the expectation values of the $\sigma_i$:
$$
	 \tau_i = Tr(\rho\sigma_i)
$$
For $M=2$ the positivity property is equivalent to $Tr\rho^2\le Tr\rho$, therefore we have $|\tau|\le 1$ and characterize the Bloch-vector-space with as a Ball with Radius $1$.
The generators of $SU(2)$ are the Pauli-matrices
$$
 X = \begin{bmatrix} 0 & 1 \\
                    1 & 0
        \end{bmatrix},~
    ~Y = \begin{bmatrix} 0 & -i \\
                    i & 0
         \end{bmatrix},~
    ~Z = \begin{bmatrix} 1 & 0 \\
                    0 & -1
        \end{bmatrix}
$$
For $M\ge3$ there are bloch vectors which do not correspond to a positive semi-definite matrix.
The space spanned by the bloch-vectors is therefore not a solid ball with radius $1$.
The generators of  $SU(3)$ are the Gell-Mann-matrices. \todo{tex gellmann} \todo{more math on qudits}\\
In quantum computing, we mostly deal with $N$ 2-level systems, the compositie space of which is $H = H_1 \otimes H_2 \otimes \ldots \otimes H_N$
In this space, there are states $\rho$ which can not be expressed through a tensor product of states in the subsystems $\rho = \rho_1\otimes\rho_2\ldots\otimes\rho_N$.
We call these states entangled states.
States which can be expressed as such are called seperable or product states.
\todo{quantum channels}
\todo{bigO notation}
