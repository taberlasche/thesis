A $k$-local-Hamiltonian is a hermitian matrix acting on $N$ qudits, which can be written as a sum of Hamiltonians where each acts on at most $k$ qudits.
Specifically, here we look at  $2$-local-Hamiltonians on qubits of the form \[
H = H_1+H_2
.\]
where \[
	H_1 = \sum_{j=1}^{3n} D_jP_j, ~ H_2  = \sum_{i,j=1}^{3n} C_{i,j}P_iP_j
.\]
with the Pauli-operators \[
	P_{3a-2}=X_a, ~ P_{3a-1}=Y_a, ~ P_{3a}=Z_a
.\]
Finding the maximal eigenvalue of such a Hamiltonian is relevant to condensed matter and chemistry.
This is equivalent to finding the minimal eigenvalue, because $\lambda_{max}(-H)=\lambda_{min}(H)$. \cite{gharibian19}
We call this the local-Hamiltonian problem, and it is correlated to finding the energy of a system at low temeratures.
Since the quantum state achieving this optimal value might be an entangled state which might not be computable in polynomial time, we are interested in finding the product state that achieves the best approximation. \\
As an elementary example, let us look at a two qubit Hamiltonian: \[
H=X_1X_2+Z_1Z_2
.\]
The state achieving the maximal eigenvalue $\lambda_{max}=2$ is the bell state $\ket{bell}=\frac{\left\bra{00}+\bra{11}\right}{\sqrt{2}}$.
This is a maximally entangled state.
To find out the product state which approximates this the best, look at a general product state and maximize the overlap.
\[
	\ket{\psi}=\ket{\psi_1}\bigotimes\ket{\psi_2}=\left(a_1\ket{0}+b_1\ket{1}\right)\bigotimes\left(a_2\ket{0}+b_2\ket{1}\right)
.\] with $a_1^2+b_1^2=a_2+b_2^2=1$
\[
	\max_{\psi_1,\psi_2}\left(\left|\braket{bell}{\psi}\right|^2\right)=\max\left(\left|\frac{1}{\sqrt{2}\left(a_1a_2+b_1b_2\right)}\right|^2\right)=\frac{1}{2}
.\]
With either $a_1=a_2=1$ and $b_1=b_2=0$ or $b_1=b_2=1$ and $a_1=a_2=0$.
Therefore, the product states with the maximal overlap are $\ket{00}$ and $\ket{11}$ with maximal eigenvalue $\lambda_{sep}=1$, the approximation ratio being  $frac{\lambda_{sep}}{\lambda_{max}} = 0.5 $\\
The local-Hamiltonian problem is equivalent to constraint satisfaction problems in classical computational theory.\todo{why?}
It is in the complexity class QMA, which is the quantum analogue to the NP complexity class.\todo{more on complexity}
It is QMA-complete, meaning that, additionally to being in the class itself, every problem in QMA can be reduced to the local-Hamiltonian problem.\cite{kempe06}
Reduction means that for predicates $L_1$ and $L_2$ there is a polynomial $f$, such that $L_1(x)=L_2(f(x))$.
We say that $f$ reduces  $L_1$ to $L_2$ polynomially.\cite{kitaev02}
It is intructive to think about finding the maximal (or minimal) eigenvalue of such a Hamiltonian as equivalent to the weighted max-cut-problem.
Given a Graph $G=(V,E)$, we think about the spin-sites as our vertices, and $C_{i,j}$ as our weighted edges.
The task now is to find a maximum cut, which is in NP.\todo{maxcut picture?}
This means we cut the graph into two sets of vertices, such that the sum of weights that we cut through is maximized.
The most successful classical approximation algorithm for this problem by Goemans and Williamson uses semidefinite programming and randomized rounding. \cite{goemans95}
The algorithm discussed here parallelizes this for the quantum case.
It is based on relaxation of a semidefinite program.
