I will now discuss the results of my implementation of the algorithm and the implemented Hamiltonians.
The Ising model with a transverse field has physical relevance because it can be used to study order disorder ferroelectrics with a tunneling effect.


As an elementary example, let us look at a two qubit Hamiltonian: \[
H=X_1X_2+Z_1Z_2
.\]
The state achieving the maximal eigenvalue $\lambda_{max}=2$ is the EPR-state $\ket{EPR}=\frac{\bra{00}+\bra{11}}{\sqrt{2}}$ .
This is a maximally entangled state.
To find out the product state which approximates this the best, look at a general product state and maximize the overlap.
\[
	\ket{\psi}=\ket{\psi_1}\otimes\ket{\psi_2}=\left(a_1\ket{0}+b_1\ket{1}\right)\otimes\left(a_2\ket{0}+b_2\ket{1}\right)
.\] with $a_1^2+b_1^2=a_2+b_2^2=1$
\[
\max_{\psi_1,\psi_2}\left(\left|\braket{\text{EPR}}{\psi}\right|^2\right)=\max\left(\left|\frac{1}{\sqrt{2}}\left(a_1a_2+b_1b_2\right)\right|^2\right)=\frac{1}{2}
.\]
With either $a_1=a_2=1$ and $b_1=b_2=0$ or $b_1=b_2=1$ and $a_1=a_2=0$.
Therefore, the product states with the maximal overlap are $\ket{00}$ and $\ket{11}$ with maximal eigenvalue $\lambda_{sep}=1$, the approximation ratio being  $\frac{\lambda_{sep}}{\lambda_{max}} = 0.5 $\\
