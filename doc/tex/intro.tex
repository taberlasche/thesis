Approximating quantum systems with product states is a relevant method to several fields.
In chemistry, it is called the Hartree-Fock method, in quantum many-body physics we call it mean-field theory.
It is successful in approximating a solution to the Schrödinger equation of atoms, molecules and nanostructures \cite{abdulsattar12}.\\
In this thesis I will give an introduction to a paper by Bravyi \emph{et al.} \cite{bravyi19}, introducing the reader to the basic concepts.
The paper demonstrates an efficient algorithm for approximating the ground state energy of traceless $2$-local Hamiltonians.
Finding the ground state of such a system is also highly relevant to classical and quantum complexity theory, since it generalizes the max-$k$-sat problem, where we are given a set of Boolean constraints on $k$ variables each and try to satisfy as many constraints as possible.
In this sense, the topic of this thesis is a beautiful intersection, since in its theory it combines physics, mathematics and computer science, while in its application it is useful for multiple subfields of chemistry, physics and computer science.
