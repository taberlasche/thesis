Finding efficient algorithms to compute product state approximations for quantum many body systems plays a key role in investigating physical models.
The general problem of product state approximation has been studied for nearly a century now and has applications in a remarkable variety of fields, including quantum many body physics, chemistry and condensed matter physics.
Finding tractable methods to realize good approximations to groundstate configurations of quantum systems has attracted much attention, since it is the quantum analogue of constraint satisfaction where we are given a set of Boolean constraints on $k$ variables each and try to satisfy as many constraints as possible.
The topic of this thesis presents a beautiful intersection of several fields, since its theory combines physics, mathematics and computer science, and has promising benefits for different subfields of chemistry and physics.\\
Using classical algorithms to approximate the maximal energy approximation is a useful method which has recently been extensively studied, see for example \cite{gharibian19, gharibian12, kempe06, brandao14, harrow17,bravyi19,anshu20}.
The objective of this thesis is to provide an introduction to the paper by Bravyi \emph{et al.} \cite{bravyi19}, which demonstrates an efficient algorithm for approximating the groundstate energy of traceless $2$-local Hamiltonians, presenting an overview of the basic concepts and examining the mathematical reasoning behind them.
I will also illustrate the challenge of generalizing such an algorithm to three-level systems, and propose a rounding scheme that parallelizes the one used for qubits.\\
The thesis is organized as follows:
In the second chapter, an introduction to quantum information theory is given.
I then proceed to provide an overview of methods from classical computing which are needed to understand the discussed paper.
Chapter 4 breaks down the approach of proving the results of Bravyi \emph{et al.} and presents the algorithm.
In chapter 5, I present the results of implementing the algorithm and analyze the implementation of two models of Hamiltonians.
I introduce the reader to the properties of qutrit systems in chapter 6 and propose a way of generalizing the algorithm by Bravyi \emph{et al.}
This is also tested through implementation.
All code that has been written for this thesis can be found in the appendix.
