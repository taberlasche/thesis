\todo{more introductory words talking about the gharibian19 paper and others}
The algorithm shown in the paper proceeds in the same way as the algorithm for the classical max-cut problem.
The aim is to find bounds for the maximum energy of $2$-local-Hamiltonians.\todo{better description of the aim}
Before looking at the algorithm itself, there are some prelimaries we have to look at.
\todo{thm2}
\todo{entanglement breaking channels}
\todo{appendix B proof idea?}
\todo{algo}
\todo{proof ideas lemma 2 and 3}
Hamiltonians of the kind $H = H_1+H_2$ where $H_1 = \sum_{j=1}^{3n} D_jP_j, ~ H_2  = \sum_{i,j=1}^{3n} C_{i,j}P_iP_j$ have terms, that are linear in Pauli operators.
For the theorems presented in this paper, the following lemma will enable us to reduce this Hamiltonian to a purely quadratic one.
We form a new $n+1 qubit$ Hamiltonian: \[
H'=H_2+Z_{n+1}H_1
.\]
Lemma1\todo{typesetting}
	$\lambda_{max}\left( H' \right) =\lambda_{max}\left( H \right)$. Moreover, given any $(n+1)$-qubit state  $\omega$ we can efficiently compute an $n$-qubit state $\phi$ such that \[
	\bra{\phi}H\ket{\phi} \ge \bra{\omega}H'\ket{\omega}.
	.\]
If $\omega$ is a tensor product of single qubit stabilizer states then so is $\phi$.
The idea is now, that for any $n-qubit$ Hamiltonian with linear terms, there is a purely quadratic $(n+1)$-qubit Hamiltonian that has the same maximal eigenvalue and has an at best equally good product state approximation.
Therefore, the bounds that we proof for quadratic Hamiltonians are valid also for Hamiltonians with linear terms.
This enables us to set $H_1=0$
The proof idea is that all eigenvalues of $H'$ are either eigenvalues of $H_2-H_1$ or $H_1+H_2$, and that $H_2-H_1$ can be obtained from $H_1+H_2$ by operations that conserve the spectrum.\todo{compute that? including time reversal?}
We can then choose the product state $\ket{\phi}$ according to $\ket{\omega}$, such that its eigenvalues will always be at least equal. \todo{show computation?} \\
The last statement in the lemma references an elegant concept that is very useful to quantum error correction.
We say an operator $A$ stabilizes a state $\ket{\psi}$ if $A\ket{\psi}=\ket{\psi}$.
Conversely, a state is called a stabilizer state of an operator, if it is in its $+1$-eigenspace.
For practicality, we look at operators from the $n$-qubit Pauli group.
This is favorable because they are unitary and their eigenvalues  $(\pm 1)$ differ significantly from another, such that we can easily perform phase estimations to find out the eigenvalue.
If we are given a set of operators $S=\{A,B,C\ldots\}$, we know that any errors (which are also from the Pauli group) either commute or anticommute with elements in $S$.
One can correct any error $E$ that anticommutes with $S$, and if the error lies in $S$ it is correctable if they commute with $S$.\cite{gottesman97}
We say an operator commutes with a group, or is in the normalizer of the group, if for some $A,B\in S$: $EA=BE$ with possibly $A\neq B$.
This criterion is very easy to check and gives us a useful mathematical toolbox. \\


\ldots
In our case, the semidefinite program is:
\begin{align*}
max ~ tr\left( CM \right)\\
s.t. M_{i,i} = 1 \\
M \ge 0
\end{align*}\todo{why ~exactly~ is this equivalent to program shown earlier}
where M is a real symmetric matrix.
This is without loss of generality, as it does not change the outcome of the objective function. \todo{thats not the only reason we can do this}
The ideal solution $M$ is connected to our state in the following way:\cite{gharibian19}
\[
	M_{i,j} = tr\left( \rho P_{i}P_j \right) ~ i,j=1\ldots 3n
.\]
From this perspective. the constraints can be understood in the following way:
$$M_{i,i}=tr\left( \rho P_i P_i\right) = tr\left( \rho\right) = 1  $$ since $P_iP_i=1$
$$M\ge 0\leftrightarrow x^TMx=tr\left(\rho\left(\sum_{i}^{3n} x_iP_i\right)\left(\sum_{j}^{3n} x_jP_j\right)\right)=tr\left(\rho X^2\right)\ge 0$$ where $X=\sum_{i}^{3n} x_iP_i$ and since $X^2,\rho\ge 0$
\todo{typesetting is ugly here}
