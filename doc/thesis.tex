\documentclass[encoding=utf8,british]{tumphthesis}
% \documentclass[pstricks,siunitx,addfonts,theorem,font=palatino,british]{tumphthesis}
% für Dissertation:
% \documentclass[encoding=utf8,british,dissertation,font=helvet]{tumphthesis}

% Die Metadaten der Abschlussarbeit (Bachelor oder Master) werden auf dem
% Deckblatt gedruckt und in dem PDF eingetragen.
\subject{Abschlussarbeit im Bachelorstudiengang Physik}
\title{Titel der Abschlussarbeit}
%\subtitle{\foreignlanguage{british}{Title in English}}
\author{Luca Göcke}
\date{31.~Juli 2011}
%\cooperators{Max-Planck-Institut für Physik}

% Auf der Rückseite des Deckblatts können Themensteller, Zweitgutachter
% und Tag der mündlichen Prüfung vermerkt werden.
\lowertitleback{Erstgutachter (Themensteller): Prof.\ A.~Kabelschacht\\
Zweitgutachter: Prof.\ S.~Preuss}

% Die Bibliographie wirde über BibLaTeX mit Biber-Backend erstellt.
% Sofern vorhanden, wird eine bib-Datei \jobname.bib (i.d.R. also der
% gleiche Name wie die Hauptdatei nur bib-Endung) automatisch eingebunden.
% Heißt sie anders oder müssen weitere Quelldateien eingebunden werden, so
% dient dazu der folgende Befehl (auch z.B. bei Verwendung von ShareLaTeX
% erforderlich, da hier \jobname auf output gesetzt wird)
 \addbibresource{thesis.bib}

\begin{document}
% Ist die Arbeit auf Englisch verfasst, hier die Sprache umschalten.
% Die Sprache muss als Klassenoption angegeben sein.
\selectlanguage{british}

\frontmatter
\maketitle
\tableofcontents

\chapter{Vorwort}
\todo{noch ausarbeiten!}

\mainmatter
\chapter{Titel des ersten Kapitels}
\section{Erster Abschnitt}


\section{Zweiter Abschnitt}
\begin{figure}
	\centering
	\includegraphics[width=\textwidth]{tumlogo}
	\caption{\label{fig:test}Test}
\end{figure}
\subsection{Unterabschnitt}
\subsubsection{Unterunterabschnitt}

\appendix
\chapter{Erstes Kapitel im Anhang}

\backmatter
\printbibliography


\end{document}
